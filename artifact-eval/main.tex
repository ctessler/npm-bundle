\documentclass[a4paper]{article}

\usepackage[margin=0.5in]{geometry}

\usepackage{xcolor}
\usepackage{listings}
\lstset{basicstyle=\small\ttfamily,
  commentstyle=\color{blue},
  stringstyle=\color{brown},
  commentstyle=\color{gray},
  %  moredelim=**[is][\color{red}]{@}{@}
  escapeinside={<@}{@>}
  }
\usepackage{hyperref}

\newcommand{\npmbundle}{\texttt{NPM-BUNDLE}}

\begin{document}

% TITLE
\title{\npmbundle{} ARTIFACT EVALUATION}
\author{Corey Tessler and Nathan Fisher}
\maketitle

\section{Introduction}

Artifact evaluation for NPM-BUNDLE is identical to the process used in
performing the research and writing. There are three components:
libsched, data generation, and document compilation. This document is
written for the evaluator that will use the Virtual Machine to
reproduce the results and (as a byproduct) the article. Although
focused on the Virtual Machine user, these instructions should be
sufficient to build the scheduling library and article from source
available on github.

Please direct any questions to Corey Tessler at \href{mailto:corey.tessler@wayne.edu}{corey.tessler@wayne.edu}.

\subsection{Abbreviated Instructions}

These abbreviated instructions use the pre-built libsched library to
produce a new version of the results as quickly as possible. More
detailed instructions are given in the following section.

\begin{description}
  \item[Virtual Machine] \hfill \\
    Long URL: \href{https://waynestateprod-my.sharepoint.com/:f:/g/personal/fh3227\_wayne\_edu/EprJwQXLTvBFo6n5XGh5-PUBBIL87kMSCfwKUTH8vlyx4w?e=Fcp1yY}{https://waynestateprod-my.sharepoint.com/:f:/g/personal/fh3227\_wayne\_edu/EprJwQXLTvBFo6n5XGh5-PUBBIL87kMSCfwKUTH8vlyx4w?e=Fcp1yY} \\
    Short URL: \href{https://bit.ly/2VwM3eF}{https://bit.ly/2VwM3eF}
    \hfill \\
    User/Pass: bundle/bundle \\

    Download the latest NPM-BUNDLE-${<}$VER${>}$.zip image, compatible
    with VirtualBox. Give the Virtual Machine a minimum of 1GB of ram
    (the more the better). Connect via the VirtualBox graphical
    console or SSH.
    
  \item[Generate Data and Graphs] \hfill \\
    \begin{lstlisting}[language=bash]
# Update the PATH for the default installation
bundle@osboxes:~/npm-bundle/libsched$ . bin/npm-paths.sh

# Build the graphs (will take 72 hours or more)
bundle@osboxes:~$ cd npm-bundle/npm-bundle/eval/
bundle@osboxes:~/npm-bundle/npm-bundle/eval$ make sanitary && \
    make run && \
    make data && \
    make plot
    \end{lstlisting}
  \item[Verify the Products] \hfill \\
    \begin{lstlisting}[language=bash]
# All possible graphs
bundle@osboxes:~/npm-bundle/npm-bundle/eval$ evince plot/all-plots.pdf
    \end{lstlisting}
  \item[Generate the Article] \hfill \\
    \begin{lstlisting}[language=bash]
# Generates the latest version with the fresh plots
bundle@osboxes:~$ cd ~/npm-bundle/npm-bundle/
bundle@osboxes:~/npm-bundle/npm-bundle$ make
bundle@osboxes:~/npm-bundle/npm-bundle$ evince npm-bundle.pdf
    \end{lstlisting}
\end{description}

\clearpage

\section{Complete Instructions}

In this section, the complete instructions for building all of the
evaluation components from source are given. An evaluator that wishes
to use the Virtual Machine should skip to subsection~\ref{sec:running}.

There are three parts, the first builds libsched. The
libsched library provides the task set generation and schedulability
tests used in the work. The second part generates the data and plots
used in the work. The final part generates the article. These
instructions are identical to those for the Virtual Machine, which is
built from an up-to-date Ubuntuu 18.10 (as of 2019/04/09).

\subsection{libsched}

The library is available on github with release tag v1.0 at the
following URL:

\href{https://github.com/ctessler/libsched/releases/tag/v1.0}{https://github.com/ctessler/libsched/releases/tag/v1.0} \\

{\noindent}Begin by installing the dependencies for libsched.

\begin{lstlisting}[language=bash]
  bundle@osboxes:~$ sudo apt-get install valgrind libcunit1-dev libgsl0-dev
\end{lstlisting}

{\noindent}Check out libsched
\begin{lstlisting}[language=bash]
  bundle@osboxes:~$ mkdir npm-bundle
  bundle@osboxes:~/npm-bundle$ git clone https://github.com/ctessler/libsched.git
\end{lstlisting}

{\noindent}Build libsched
\begin{lstlisting}[language=bash]
  bundle@osboxes:~/npm-bundle/libsched$ make
\end{lstlisting}

{\noindent}Upon success, there will be several products in the bin directory.
\begin{lstlisting}[language=bash]
  bundle@osboxes:~/npm-bundle/libsched$ ls bin
  maxchunks  ts-deadline-bb  ts-gen    ts-gentp-forwcet  ts-merge  unittest
  tpj        ts-divide       ts-gentp  ts-gf             ts-print  uunifast
\end{lstlisting}

{\noindent}Verify the tpj schedulability test is operational by
\begin{lstlisting}[language=bash]
  bundle@osboxes:~/npm-bundle/libsched$ bin/tpj -s ex/one_split.ts
  Task set file: ex/one_split.ts
  Task Set:
  1: (p:  10, d:   5, m: 1) [u:0.100, q:0, t.1]  wcet{  1}
  2: (p:  30, d:  30, m: 2) [u:0.200, q:0, t.2]  wcet{  3,   6}

  After assigning non-preemptive chunks
  1: (p:  10, d:   5, m: 1) [u:0.100, q:1, t.1]  wcet{  1}
  2: (p:  30, d:  30, m: 1) [u:0.100, q:3, t.2]  wcet{  3}
  3: (p:  30, d:  30, m: 1) [u:0.100, q:3, t.2]  wcet{  3}
  -------------------------------------------------
  Utilization: 0.3000, T*: 30, Feasible: Yes
\end{lstlisting}

{\noindent}\textbf{Add the binary directory to the working PATH}
\begin{lstlisting}[language=bash]
  bundle@osboxes:~/npm-bundle/libsched/bin$ pwd
      /home/bundle/npm-bundle/libsched/bin
  bundle@osboxes:~/npm-bundle/libsched/bin$ export \
      PATH=${PATH}:/home/bundle/npm-bundle/libsched/bin
\end{lstlisting}

\subsection{Generating Data and Plots}

The evaluation is incorporated with the latex structure of the
article. Clone the most recent version from the following URL

\href{http://github.com/ctessler/npm-bundle.git}{http://github.com/ctessler/npm-bundle.git}

\begin{lstlisting}[language=bash]
  bundle@osboxes:~$ cd npm-bundle/
  bundle@osboxes:~/npm-bundle$ git clone https://github.com/ctessler/npm-bundle.git
\end{lstlisting}

\subsubsection{Running the evaluation}\label{sec:running}

The binaries from libsched and the shell scripts of the evaluation
must be in the current path for the evaluation to run. Make sure to
add them by one of the following methods (the second method is only
available to those using the Virtual Machine).

\begin{lstlisting}[language=bash]
  # Option 1
  bundle@osboxes:~$ export PATH=${PATH}:~/npm-bundle/libsched/bin
  bundle@osboxes:~$ export PATH=${PATH}:~/npm-bundle/npm-bundle/eval/bin
\end{lstlisting}

\begin{lstlisting}[language=bash]
  # Option 2 (Only available on the Virtual Machine)
  bundle@osboxes:~$ . bin/npm-paths.sh
\end{lstlisting}

{\noindent}Verify the PATH is set correctly
\begin{lstlisting}[language=bash]
  bundle@osboxes:~$ echo $PATH
    /home/bundle/bin:/usr/local/sbin:/usr/local/bin:/usr/sbin:/usr/bin:
    /sbin:/bin:/usr/games:/usr/local/games:/snap/bin:
    <@\textcolor{blue}{/home/bundle/npm-bundle/libsched/bin:/home/bundle/npm-bundle/npm-bundle/eval/bin}@>
\end{lstlisting}


{\noindent}The evaluation runs from within the \texttt{npm-bundle.git/eval}
directory, there are two environment variables that control the
performance and scale of the run. \texttt{COUNT} impacts the scale,
the default is 1,000. \texttt{J} sets a limit on the number of
processes that can be forked during a run.\\


{\noindent}On the virtual machine, a sample run has been completed for
\texttt{COUNT=100} and \texttt{J=2} by issuing the following commands:

\begin{lstlisting}[language=bash]
  bundle@osboxes:~$ cd npm-bundle/npm-bundle/eval/
  bundle@osboxes:~/npm-bundle/npm-bundle/eval$ COUNT=100 J=2 make run
  bundle@osboxes:~/npm-bundle/npm-bundle/eval$ COUNT=100 J=2 make data
  bundle@osboxes:~/npm-bundle/npm-bundle/eval$ COUNT=100 J=2 make plot
\end{lstlisting}

{\noindent}The results can be found in the plot dircetory
\begin{lstlisting}[language=bash]
  bundle@osboxes:~$ evince npm-bundle/npm-bundle/eval/plot/all-plots.pdf
\end{lstlisting}

{\noindent}To make a new evaluation run using the default values of
\texttt{COUNT=1000} and \texttt{J=5} issue the following
command. Note, the task sets, their periods and deadlines are randomly
generated and may produce large hyperperiods which increase the number
of interval lengths that need to be checked. The larger the
\texttt{COUNT} the more likely large hyperperiods will be generated
creating ``hard'' to test task sets. The author has experienced task
sets that take more than \emph{72} hours to complete.

\begin{lstlisting}[language=bash]
  bundle@osboxes:~$ cd npm-bundle/npm-bundle/eval/
  bundle@osboxes:~/npm-bundle/npm-bundle/eval$ make run
  # Go away for a few days.
\end{lstlisting}

{\noindent}When completed, a new run directory will contain the
numerical results. To generate plots, the run must be promoted to
data. 

\begin{lstlisting}[language=bash]
  bundle@osboxes:~/npm-bundle/npm-bundle/eval$ make data
\end{lstlisting}

{\noindent}Plots are made from the current data set.
\begin{lstlisting}[language=bash]
  bundle@osboxes:~/npm-bundle/npm-bundle/eval$ make plot
\end{lstlisting}

{\noindent}Verify the plots have been generated.
\begin{lstlisting}[language=bash]
  bundle@osboxes:~/npm-bundle/npm-bundle/eval$ evince plot/all-plots.pdf
\end{lstlisting}


\subsection{Generating the Article}

To generate the article the plots must be linked into the build
structure (this step has already been performed on the Virtual
Machine).

\begin{lstlisting}[language=bash]
  bundle@osboxes:~$ cd npm-bundle/npm-bundle/tex/
  bundle@osboxes:~/npm-bundle/npm-bundle/tex$ ln -s ../eval/plot plot
\end{lstlisting}

{\noindent}Make the article.
\begin{lstlisting}[language=bash]
  bundle@osboxes:~$ cd npm-bundle/npm-bundle/
  bundle@osboxes:~/npm-bundle/npm-bundle$ make
\end{lstlisting}

{\noindent}Verify the result.
\begin{lstlisting}[language=bash]
  bundle@osboxes:~/npm-bundle/npm-bundle$ evince npm-bundle.pdf
\end{lstlisting}

\subsection{Success!}

Having created the all-plots.pdf and npm-bundle.pdf documents, the
complete set of artifacts have been reproduced. Those that wish to make
changes to the schedulability tests should examine the README.md
within the libsched directory. Those that wish to augment the
evaluation of this article should begin with the README.txt in the
eval/ directory of npm-bundle.git.

\end{document}
