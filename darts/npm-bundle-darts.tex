\documentclass[a4paper,USenglish,english]{darts-v2018}
%This is a template for producing DARTS artifact descriptions. 
%for A4 paper format use option "a4paper", for US-letter use option "letterpaper"
%for british hyphenation rules use option "UKenglish", for american hyphenation rules use option "USenglish"
% for section-numbered lemmas etc., use "numberwithinsect"
 
\usepackage{microtype}%if unwanted, comment out or use option "draft"

%\graphicspath{{./graphics/}}%helpful if your graphic files are in another directory

%\nolinenumbers to disable line numbers

\bibliographystyle{plainurl}% the recommended bibstyle

% Commands for artifact descriptions
% Written by Camil Demetrescu and Erik Ernst
% April 8, 2014

\newenvironment{scope}{\section{Scope}}{}
\newenvironment{content}{\section{Content}}{}
\newenvironment{getting}{\section{Getting the artifact} The artifact 
endorsed by the Artifact Evaluation Committee is available free of 
charge on the Dagstuhl Research Online Publication Server (DROPS).}{}
\newenvironment{platforms}{\section{Tested platforms}}{}
\newcommand{\license}[1]{{\section{License}#1}}
\newcommand{\mdsum}[1]{{\section{MD5 sum of the artifact}#1}}
\newcommand{\artifactsize}[1]{{\section{Size of the artifact}#1}}

\newcommand{\orcid}[1]{\url{http://orcid.org/#1}}
\newcommand{\email}[1]{\href{mailto:#1}{\texttt{#1}}}

% Author macros::begin %%%%%%%%%%%%%%%%%%%%%%%%%%%%%%%%%%%%%%%%%%%%%%%%
\title{\texttt{NPM-BUNDLE}: Non-Preemptive Multitask Scheduling for
  Jobs with \texttt{BUNDLE}-based Thread-Level Scheduling (Artifact)}

%optional, in case that the title is too long; the running title
%should fit into the top page column
\titlerunning{\texttt{NPM-BUNDLE} (Artifact)}

% ARTIFACT: Authors may not be exactly the same as the related
% scholarly paper, e.g., you may want to include authors who
% contributed to the preparation of the artifact, but not to the
% companion paper 

\author{Corey Tessler}{
  Wayne State University,
  {Detroit, Michigan, United States}
}{corey.tessler@wayne.edu}{}{}
%mandatory please use full name; only 1 author per \author macro; first two
%parameters are mandatory, other parameters can be empty. 

\author{Nathan Fisher}{
  Wayne State Univeristy, {Detroit, Michigan, United States}
}{fishern@wayne.edu}{}{}

\authorrunning{C. Tessler and N. Fisher}
%mandatory. First: Use abbreviated first/middle names. Second (only in
%severe cases): Use first author plus 'et. al.' 

\Copyright{Corey Tessler and Nathan Fisher}
%mandatory, please use full first names. DARTS license for the
%artifact description is "CC-BY"; %http://creativecommons.org/licenses/by/3.0/ 

\subjclass{
  \ccsdesc{Computer systems organization~Real-time systems}
  \ccsdesc{Software and its engineering~Real-time schedulability}
}
% mandatory: Please choose ACM 2012 classifications from
% https://www.acm.org/publications/class-2012 or
% https://dl.acm.org/ccs/ccs_flat.cfm . E.g., cite as "General and
% reference $\rightarrow$ General literature" or \ccsdesc[100]{General
% and reference~General literature}.  

\keywords{  
  Scheduling algorithms, Cache Memory, Multi-threading,
  Static Analysis
}% mandatory: Please provide 1-5 keywords
% Author macros::end %%%%%%%%%%%%%%%%%%%%%%%%%%%%%%%%%%%%%%%%%%%%%%%%%

% Please provide information to the related scholarly information
\RelatedArticle{Corey Tessler and Nathan Fisher,
  ``\texttt{NPM-BUNDLE}: Non-Preemptive Multitask Scheduling for Jobs
  with \texttt{BUNDLE}-based Thread-Level Scheduling'',
  in Proceedings of the 31st Euromicro Conference on Real-Time Systems (ECRTS’19).}

%Editor-only macros:: begin (do not touch as author)%%%%%%%%%%%%%%%%%%%%%%%%%%%%%%%%%%
\Volume{3}
\Issue{2}
\Article{1}
\RelatedConference{42nd Conference on Very Important Topics (CVIT 2016), December 24--27, 2016, Little Whinging, United Kingdom}
% Editor-only macros::end %%%%%%%%%%%%%%%%%%%%%%%%%%%%%%%%%%%%%%%%%%%%%%%

\begin{document}

\newcommand{\bundlep}{\texttt{BUNDLEP}}
\newcommand{\bundle}{\texttt{BUNDLE}}
\newcommand{\npmbundle}{\texttt{NPM-BUNDLE}}
\newcommand{\tpj}{\text{\sc{tpj}}}

\maketitle

\begin{abstract}
  The \bundle{} and \bundlep{} scheduling algorithms are cache-cognizant
  thread-level scheduling algorithms and associated worst case
  execution time and cache overhead (WCETO) techniques for hard
  real-time multi-threaded tasks. The \texttt{BUNDLE}-based approaches
  utilize the inter-thread cache benefit to reduce WCETO values for 
  jobs. Currently, the \texttt{BUNDLE}-based approaches are limited to
  scheduling a single task. This work aims to expand the applicability
  of \texttt{BUNDLE}-based scheduling to multiple task multi-threaded
  task sets.

  \texttt{BUNDLE}-based scheduling leverages knowledge of potential
  cache conflicts to selectively preempt one thread in favor of
  another from the same job. This thread-level preemption is a
  requirement for the run-time behavior and WCETO calculation to
  receive the benefit of \texttt{BUNDLE}-based approaches. This work
  proposes scheduling \texttt{BUNDLE}-based jobs non-preemptively
  according to the earliest deadline first (EDF) policy. Jobs are
  forbidden from preempting one another, while threads within a job
  are allowed to preempt other threads.

  An accompanying schedulability test is provided, named Threads Per
  Job (\tpj{}). \tpj{} is a novel schedulability test, input is a task
  set specification which may be transformed (under certain
  restrictions); dividing threads among tasks in an effort to find
  a feasible task set. Enhanced by the flexibility to transform task
  sets and taking advantage of the inter-thread cache benefit, the
  evaluation shows \tpj{} scheduling task sets fully preemptive EDF
  cannot. 
\end{abstract}

% ARTIFACT: please stick to the structure of 7 sections provided below

% ARTIFACT: section on the scope of the artifact (what claims of the
% paper are intended to be backed by this artifact?) 
\begin{scope}
  The artifacts for non-preemptive multi-task \bundle{} (\npmbundle{})
allow an independent party to recreate and expand upon the results
presented in the research. The primary focus is reproduction of the
ten graphs that summarize the schedulability ratios of preemptive EDF,
non-preemptive EDF, and the proposed Threads Per Job (\tpj{})
algorithms.

  Data for the graphs is supplied by the creation and analysis of
synthetic task sets. Synthetic task sets and their analysis depend on
two components 1.) the libsched library 2.) a framework utilizing
libsched. Both of these components are pre-built and configured in the
supplied virtual machine of the artifact.

  The virtual machine image and accompanying instructions provide
direction for reproducing the results presented in the research, as
well as additional results and methods for tailoring the data set size
and parameters to suit subsequent research.

\end{scope}

% ARTIFACT: section on the contents of the artifact (code, data, etc.)
\begin{content}
When the artifact package is expanded, it includes:
\begin{itemize}
\item \texttt{NPM-BUNDLE-01.zip}: A virtual machine image compatible
  with Virtual Box.
\item \texttt{NPM-BUNDLE-artifact-eval.pdf}: A document describing the
  use of the environment present on the virtual machine to generate
  synthetic tasks and analyze their results. Additionally, this
  document provides instructions on 1.) how to acquire and install the
  libsched library 2.) modify parameters of data generation used by
  the framework.
\end{itemize}
\end{content}

% ARTIFACT: section containing links to sites holding the
% latest version of the code/data, if any
\begin{getting}
% leave empty if the artifact is only available on the DROPS server.
% otherwise, provide links to the latest version of the artifact (e.g., on github)
  In addition, the artifact is also available as \texttt{artifact.tgz}
  at:
  \begin{itemize}
    \item Long URL: \url{https://waynestateprod-my.sharepoint.com/:f:/g/personal/fh3227\_wayne\_edu/EprJwQXLTvBFo6n5XGh5-PUBBIL87kMSCfwKUTH8vlyx4w?e=Fcp1yY} 
    \item Short URL: \url{https://bit.ly/2VwM3eF}
  \end{itemize}
\end{getting}

% ARTIFACT: section specifying the platforms on which the artifact is known to
% work, including requirements beyond the operating system such as large
% amounts of memory or many processor cores
\begin{platforms}
  The virtual machine is known to operate correctly on a host system
  with an Intel(R) Core i5-4690K at 3.50 GHz with 16 gigabytes of
  memory. Two cores and one gigabyte of memory was dedicated to the
  virtual machine guest. With these resources the results are generated in the
  range of ${[2, 72]}$ hours. The variance is due to the nature of synthetic task
  parameters being generated by pseudo-random algorithms.
\end{platforms}

% ARTIFACT: section specifying the license under which the artifact is
% made available
\license{The artifact is available under license Creative Commons CC-BY.}

% ARTIFACT: section specifying the md5 sum of the artifact master file
% uploaded to the Dagstuhl Research Online Publication Server, enabling 
% downloaders to check that the file is the expected version and suffered 
% no damage during download.
\mdsum{9b89f75e677bd37728faac2b920ae240}

% ARTIFACT: section specifying the size of the artifact master file uploaded
% to the Dagstuhl Research Online Publication Server
\artifactsize{5.4 GiB}

\end{document}
