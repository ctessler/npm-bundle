\section{Introduction}

Hard real-time multi-threaded task systems which incorporate cache
memory, must account for the variation in execution time and cache
related preemption delays found in single-threaded task systems. For
multi-threaded task systems, the complexity of cache interactions is
increased due to thread-level cache interference and preemptions.
Worst-case execution time (WCET) and schedulability analysis of hard
real-time multi-threaded tasks commonly treat threads
independently~\cite{Pellizzoni:2011} or utilize cache management
techniques~\cite{Ward:2013} to limit the cache interference. 

Analysis techniques focusing on independent treatment or
limiting of cache interference exclude the possible benefit of
caches. Multi-threaded tasks may benefit from caches. By virtue of
sharing the same address space one thread of a task may cache values
on behalf of another reducing the total execution time to complete
both. This positive effect is referred to as the
\emph{inter-thread cache benefit}~\cite{Tessler:2016}. 

Currently, only the \bundle{}~\cite{Tessler:2016} and
\bundlep{}~\cite{Tessler:2018} analysis techniques and cache
congnizant thread-level scheduling algorithms incorporate the
inter-thread cache benefit into WCET and schedulability
analysis. These \texttt{BUNDLE}-based approaches are currently limited
to a \underline{single} multi-threaded task. The primary focus of this
work is to provide a scheduling algorithm and schedulability test for
multi-threaded task sets with multiple tasks, where individual jobs
utilize \texttt{BUNDLE}-based scheduling. As the first scheduling
algorithm to incorporate \texttt{BUNDLE}-based thread-level
scheduling, a non-preemptive algorithm was chosen to avoid necessary
modifications to \bundle{} and \bundlep{}. Non-preemptive EDF was
selected as the task-level scheduler, as the proposed schedulability
test presented in Section~\ref{sec:schedulability} is based upon
Baruah's limited-preemption for EDF~\cite{Baruah:2005} algorithm. 

An additional consideration is made for alternative approaches and
the unforeseen benefits to schedulability of thread-level schedulers
of non-preemptive multi-threaded jobs. If the WCET of jobs can be
expressed as a strictly increasing discrete concave function of the
number of threads per job, the schedulability test developed for this
work applies without modification to the \texttt{BUNDLE}-based
approaches or non-preemptive EDF scheduling. 

In the following sections, the key contributions are:
\begin{enumerate}
  \item A model of hard real-time multi-threaded tasks which is
    compatible with existing single-threaded models, where tasks sets
    may be transformed through division of threads.
  \item A schedulability test named Threads Per Job (\tpj{}) that provides a
    schedulability result and transformed feasible task set if the
    specified task set could not be scheduled non-preemptively.
  \item Proof of \tpj{}'s optimality with respect to non-preemptive
    multi-threaded feasibility.
  \item An improvement to Baruah's~\cite{Baruah:2005}
    non-preemptive chunk algorithm, increasing chunk sizes.
  \item An evaluation of over 500,000 task sets, comparing the
    schedulability ratio of \tpj{} to those of non-preemptive and
    (limited) preemptive EDF, with an accompanying implementation available for
    download.~\cite{NPM-Artifact:2019}.
\end{enumerate}

These contributions are presented following the related
research of Section~\ref{sec:related}. Section~\ref{sec:model}
introduces the proposed model, application of non-preemptive EDF
scheduling for thread-level schedulers, and the requirements of task
transformation. Section~\ref{sec:schedulability} introduces then
improves upon the non-preemptive chunk algorithm~\cite{Baruah:2005},
followed by the \tpj{} schedulability algorithm and proof of
feasibility. Section~\ref{sec:eval} compares the schedulability ratio
of \tpj{} to other non-preemptive and preemptive scheduling
algorithms, before concluding with Section~\ref{sec:conclusion}. 

  

