Scheduling and schedulability analysis proposed in this work relies upon
a relationship between the number of threads scheduled per
multi-threaded job and the WCET of the job executed
non-preemptively. To clarify, the scheduling mechanism proposed in
this work precludes preemptions between jobs of different tasks. For
threads within a job of a task, a thread-level scheduler may execute
threads preemptively. Figure~\ref{fig:preemptivity} illustrates the
scheduling behavior.

\begin{wrapfigure}[7]{i}{.35\linewidth}
  \centering
  \input{../../svg/preemptivity.pdf_tex}
  \caption{Scheduling Behavior}
  \label{fig:preemptivity}
\end{wrapfigure}

In Figure~\ref{fig:preemptivity}, at ${t = 1}$ a job of \task{2} is
released. The job of \task{2} cannot be preempted by the job of \task{1} 
released at ${t = 5}$. During the execution of \task{2}, the two
threads (given distinct colors) may preempt one another according to
the thread-level scheduler, at ${t = 8}$ for instance. Thread-level
scheduling and preemption decisions are not prescribed by this
work. The thread-level scheduling policies of \task{1} and \task{2}
are independent of the non-preemptive task-level scheduling of
non-preemptive EDF used in this work. 
