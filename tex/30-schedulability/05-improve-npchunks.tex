\subsection{Improving the Non-Preemptive Chunk Size}
\label{sec:improve-np-chunk}

From the description of \npchunks{} in~\cite{Baruah:2005}, there is
an opportunity to improve the available slack for each of the ${k}$
deadlines considered. Alogrithm~\ref{alg:np-chunks} is pessimistic in
the amount of available slack at any deadline ${D_k}$. To illustrate,
consider the task set and intermediate values described by
Figure~\ref{fig:exbad}. 

\begin{figure}[H]
  \begin{tabular}{cc}
    \begin{tabular}{|c|c|c|c|c|}
      \hline
      ${i}$ & \period{i} & \deadline{i} & \taskthreads{i} &
      \wcet{i}{\taskthreads{i}} \\
      \hline
      \task{0} & 4 & 2 & 1 & 1 \\
      \task{1} & 3 & 3 & 1 & 1 \\
      \task{2} & 3 & 3 & 1 & 1 \\
      \hline
    \end{tabular}
    &
    \begin{tabular}{|c||c|c|c|c||c|}
      \hline
      \hp & \Deadline{k} & ${\task{j}: d_j = D_k}$ & \dbf{\tasks}{\Deadline{i}}
      & \slack{\Deadline{i}} & ${q_j}$\\
      \hline
      \multirow{3}{*}{12}
      & \Deadline{1} = 2 & \task{0} & 1 & 1 & 1\\
      \cline{2-6}
      & \multirow{2}{*}{\Deadline{2} = 3} & \task{1} & 3 & 0 & \textbf{0}\\
      & & \task{2} & 3 & 0 & \textbf{0}\\      
      \hline
    \end{tabular}
  \end{tabular}
  \caption{Example Task Set ${\tasks{} = \{ \task{0}, \task{1}, \task{2} \}}$}
  \label{fig:exbad}
\end{figure}

There are three tasks in the task set of Figure~\ref{fig:exbad}, with 
utilization of approximately 0.92. For \task{0}, initialization
assigns a non-preemptive chunk of ${q_0 = 1}$ time units. By
observation, after release \task{0} may be delayed from execution by
at most one time unit or it will miss its deadline. Consequently, the
non-preemptive chunk size available to \task{1} and \task{2} is 1. As
such \npchunks{} would be expected to find ${q_0 = 1, q_1 = 1, q_2 =1}$.

Note, it is not possible for \task{0} to be blocked for 1
or more time units if both \task{1} and \task{2} execute
non-preemptively for 1 time unit each. If \task{0} is blocked for less
than 1 time unit by \task{1}, then \task{0} will be the highest
priority task when \task{1} completes (similarly for \task{2}). It is
impossible for \task{0} to be blocked 1 time unit or more by \task{1}
or \task{2}, \task{0} would have to be released at the same time
instant as \task{1} or \task{2} and \task{1} or \task{2} would have to
execute before \task{0}, since the relative deadline of \task{0} is
less than the other two, limited-preemption EDF executes \task{0}: the
task with earliest absolute deadline.

For \task{0}, ${q_0}$ is calculated as expected
${q_0 = c_0(m_0) = 1}$, by
Lines~\ref{line:npchunks-init-start}-\ref{line:npchunks-init-end} of
Algorithm~\ref{alg:np-chunks}. However, \task{1} has a non-preemptive
chunk size of ${q_1 = 0}$. The reason is Line~\ref{line:slackDk},
where \slack{\Deadline{2}} is calculated which includes the execution
demand of \task{1} and \task{2}. Slack is an upper bound on
the non-preemptive chunk size assigned to a task (in this case
\task{1}). Giving a task the available slack permits the task to
execute longer, delaying higher priority jobs from executing in the
interval by delaying them for as much time as there is slack.

By example in Figure~\ref{fig:exbad}, the available slack for \task{1}
is determined from the interval of length \Deadline{2} = 3. The
execution requirement of \task{1} and \task{2} is included
in \dbf{\tasks{}}{3} because ${\deadline{1} = \deadline{2} = 3}$. Thus
\slack{\Deadline{2}} is zero. Since \task{1}'s execution requirement
is already included, it cannot 
further interfere over the interval \Deadline{2}. Furthermore,
\task{1} must have executed some portion without being preempted or
the system would not be schedulable. Inclusion of \task{1}'s execution
requirement within the interval over which slack is calculated for is
pessimistic with respect to the non-preemptive chunk ${q_1}$ in this
specific example, and ${q_j}$ in general.


In the pseudocode implementation of \npchunks{} adopted
from~\cite{Baruah:2005}, Line~\ref{line:slackDk} calculates the
non-preemptive chunk size according Equation~\ref{eq:baruah-07}
(Equation 7 of Theorem 1 in~\cite{Baruah:2005}). Comparing
Line~\ref{line:slackDk} of Algorithm~\ref{alg:np-chunks} to
Equation~\ref{eq:baruah-07} a mismatch between the algorithm and the
infeasibility test is illuminated. 

\begin{definition}[Infeasibility Test, Equation 7, from~\cite{Baruah:2005}]
\begin{equation}
  \exists \task{j} \in \tasks{}, t \in [0, d_j) ~|~ t < q_j +
    \sum_{\substack{{i=0}\\{i \not = j}}}^{n - 1}
    \dbf{\task{i}}{$t$}
    \label{eq:baruah-07}
\end{equation}
\end{definition}

If the condition of Equation~\ref{eq:baruah-07} is
satisfied for a task set \tasks{}, the task set is unschedulable given
a limited-preemption task set with assigned non-preemptive chunks
${q}$. The interval considered in the demand of
Equation~\ref{eq:baruah-07} is over ${[0,d_j)}$. The demand used in 
Algorithm~\ref{alg:np-chunks} to calculate ${q_j}$ is over the
interval ${[0, d_j]}$. Extending the interval to include ${d_j}$
introduces the pessimism identified by the example and is not required
by Equation~\ref{eq:baruah-07}.

Figure~\ref{fig:exbad} illustrates the pessimism of
\npchunks{} found in~\cite{Baruah:2005}. The
example uses the notation of assigning non-preemptive chunks to
individual tasks from~\cite{Baruah:2005}. A 
later work~\cite{Bertogna:2010} uses a different notation,
assigning non-preemptive chunks to interval lengths for the remaining
execution of a job. The conceptual pessimism of including demand for
tasks with deadline equal to the current interval (described by
Figure~\ref{fig:exbad}) is also found in~\cite{Bertogna:2010}. 
