\subsection{Threads per Job (\tpj{}) Scheduling Algorithm}
\label{sec:tpj}

In this work, the \npchunks{} algorithm is modified for several
purposes. First, the unnecessary pessimism is removed from chunk
calculations. Second, the schedulability test is adapted to the model
used herein. Lastly, when a given assignment of tasks and threads are
infeasible, tasks are divided (when possible) to fit into their
chunks. The division process is repeated until the task set is
feasible, or no possible divisions remain and the task set is reported
as infeasible. The algorithm is named the \emph{Threads Per Job} (\tpj{})
scheduling algorithm.

A full description of \tpj{} is presented at the end of this
subsection. To reach the complete description, an intermediate
algorithm named \emph{Bigger Non-Preemptive Chunks} (\bng{}) is presented
as pseudocode in Algorithm~\ref{alg:bng}. \bng{} removes the
pessimism described in Section~\ref{sec:improve-np-chunk}. The
algorithm takes advantage of a property of the demand function 
\dbf{\tasks{}}{$t$} noted in~\cite{Baruah:2005}. 

\begin{prop}[Demand Change]\label{prop:dmnd}
Demand for a task does not change for values of ${t}$ that do not
equal an absolute deadline. In terms of the set of ordered absolute
deadlines,
${\dbf{\tasks{}}{\Deadline{i-1}} =
  \dbf{\tasks{}}{\Deadline{i}${ - \epsilon}$}}$, for
${0 < \epsilon \le (\Deadline{i} - \Deadline{i-1})}$.
\end{prop}
\input{\tex{20-bng.tex}}

Line~\ref{line:bng-improvement} of Algorithm~\ref{alg:bng} implements
the improvement of \bng{} over \npchunks{}. The non-preemptive chunk 
${q_j}$ of task \task{j} is taken from the slack of the previous
interval ${D_{k-1}}$ or the task's WCET ${c_j(m_j)}$, whichever is
smaller. The algorithm verifies the condition set by
Equation~\ref{eq:baruah-07}, selecting the correct interval
length by Property~\ref{prop:dmnd}, which precludes the inclusion of
\task{j}'s execution requirement in the interval (and other tasks with
deadline \Deadline{k}).

\begin{algorithm}[H]
  \caption{Threads-Per-Job (\tpj{})}\label{alg:tpj}
  \begin{algorithmic}[1]
    \State \slack{\Deadline{0}} ${\gets \infty}$
    \For{${k \in \{D_1, D_2, D_3, ..., \}}$} \label{line:begin-for}
      \If{${D_k > T^*(\tasks)}$} \label{line:infeasible-1}
        \State return \emph{feasible}
      \EndIf
 
      \For {${\ant{j} \in \{\task{i} \in \tasks{} ~|~ (d_i = D_k)\}}$}      \label{line:bnc-mod-start}
      \If{${\slack{\Deadline{k-1}} < \hat{c}_j(1)}$}\label{line:minq-start}
        \State return \emph{infeasible} \label{line:no-slack-infease}
        \EndIf \label{line:minq-end}
      \State ${\Partial{j} \gets \{\ant{j}\}}$
      \If{${\slack{\Deadline{k-1}} < \hat{c}_j(\hat{m}_j)}$}\label{line:rem-p1}
        \Comment Jobs must be divided
          \State ${\Partial{j} \gets \algdivide{\ant{j}}{\slack{\Deadline{k-1}}}}$
          \label{line:divide}
          \State ${\tasks{} \gets \tasks{} \setminus \ant{j}}$
          \Comment Anterior task \ant{j} is represented by \Partial{j}
          \State ${\tasks{} \gets \tasks{} \cup \Partial{j}}$
          \Comment Partial tasks include all threads of \ant{j}
      \EndIf \label{line:rem-p2}
      \For {${\task{j} \in \Partial{j}}$}
          \State ${q_j \gets \wcet{j}{m_j}}$
      \EndFor
      \EndFor \label{line:the-same}
      \label{line:bnc-mod-end}
      
      \State \slack{${D_k}$} ${\gets \min
        \left(
          \slack{${D_{k-1}}$}, D_k - \sum_{\task{i} \in \tasks}
          \dbf{\task{i}}{${D_k}$}
        \right)
      }$ \label{line:same-deadline}
      \If{\slack{${D_k}$} ${< 0}$} \label{line:infeasible-2}
        \State return \emph{infeasible} \label{line:more-demand}
      \EndIf
    \EndFor
  \end{algorithmic}
\end{algorithm}


The Threads per Job scheduling Algorithm~\ref{alg:tpj}, is a
modification of \bng{} from limited-preemption EDF (EDF-LP) scheduling to
non-preemptive EDF (EDF-NP). Input to the schedulability test is a task set
specification \tasks{}, if \tpj{} returns a \emph{feasibile} result
there exists a posterior task set which can be scheduled by
non-preemptive EDF and the posterior task set is
returned as \tasks{}. An \emph{infeasible} result indicates that
\tpj{} could not guarantee \tasks{} would be schedulable by EDF-NP for
any posterior task set. Since non-preemptive EDF is not optimal with
respect to feasibility~\cite{Buttazzo:2011}, \tpj{} is a sufficient
test but cannot be necessary.

Algorithm~\ref{alg:tpj} (\tpj{}) modifies \bng{}, the modifications
are limited to
Lines~\ref{line:bnc-mod-start}-\ref{line:bnc-mod-end}. An additional
benefit of \bng{} removing the pessimism of each ${q_j}$, is
that each ${q_j}$ can be calculated without consideration of the
current task \task{j} and the demand at ${D_k}$. Chunk values depend
on the demand of ${D_{k-1}}$ instead. This permits an
efficient implementation of \tpj{} by moving the slack calculation of
the current interval to the end of each iteration. Otherwise, if slack
were calculated earlier in each iteration, the changes to demand
resulting from Lines~\ref{line:bnc-mod-start}-\ref{line:bnc-mod-end}
would force the demand and slack of ${D_k}$ to be recalculated.

The first notable change to \bng{} is introduced on
Line~\ref{line:minq-start}, comparing the available slack to the WCET
of a single thread of \ant{j}. If there is insufficient slack to
execute just one thread of \ant{j} to completion, the task cannot be
executed non-preemptively for any number of threads and the task set
is infeasible non-preemptively.

Lines~\ref{line:rem-p1}-\ref{line:rem-p2} introduce several
subtle changes. For clarity, it is simpler to discuss
the negative case (${\slack{\Deadline{k-1}} \ge \hat{c}_j(\hat{m}_j)}$)
before the positive.
When there is sufficient slack for ${\hat{m}_j}$ threads to execute
without preemption, \ant{j} is given its full WCET
(${\hat{c}_j(\hat{m}_j)}$) as its non-preemptive chunk. In other
words, no division of \ant{j} is required and the posterior task set
\tasks{} is unchanged (with respect to \ant{j}).
Lines~\ref{line:rem-p1}-\ref{line:rem-p2} are avoided, the
algorithm progresses to the next task such that ${d_i = D_k}$.

However, in the positive case on Line~\ref{line:rem-p1} (when
${\slack{\Deadline{k-1}} < \hat{c}_j(\hat{m}_j)}$), ${\hat{m}_j}$
threads of \ant{j} cannot feasibly execute without being
preempted. Therefore, \ant{j} must be divided. The \texdivide{}
procedure creates a partial task \Partial{j} set of \ant{j}, such that
all tasks ${\task{p} \in \Partial{j}}$ will complete within the
available slack ${c_p(m_p) \le D_{k-1}}$. The posterior task set \tasks{}
has \ant{j} removed, and is replaced by the partial set \Partial{j}
maintaining the specified number of threads for \ant{j}.

For any task \ant{j}, the task is transformed into a partial task set
\Partial{j} and assigned a non-preemptive chunk only
once in the iteration where the absolute deadline \Deadline{k} is
equal to the relative deadline of the task:
${\Deadline{k} = \hat{\deadline{j}}}$. Since tasks of \tasks{} are
evaluated in strictly increasing absolute deadline order, the impact
on demand and non-preemptive chunk sizes of processing \ant{j}
exclusively impacts demand for larger intervals ${D_\ell > D_k}$ and
non-preemptive chunk sizes for tasks ${\task{\ell} \in \tasks{}}$ with
greater relative deadlines ${\deadline{\ell} > \hat{\deadline{}}_k}$.

\begin{prop}[Divisions of \ant{j} Exclusively Impacts Interval of
    Length ${t \ge \hat{\deadline{}}_j}$]
\label{prop:divisions}
Division of \ant{j} into the partial set \Partial{j}, and replacing
\ant{j} in \tasks{} with \Partial{j} will impact demand exclusively
for intervals of length ${\Deadline{k} \ge \hat{d}_j}$, 
slack of absolute deadlines ${\Deadline{k} > \hat{d}_j}$ and therefore
non-preemptive chunk values ${q_\ell}$ for tasks ${\task{\ell} \in
  \tasks{}}$ with relative deadlines ${\deadline{\ell} \ge \Deadline{k}}$

By definition of \dbf{\ant{j}}{$t$}, no task of \Partial{j} or \ant{j}
can impact the task set \tasks{} demand \dbf{\tasks{}}{$t$} when
${t < d_j}$. Thus replacing \ant{j} in \tasks{}, only affects the
demand of intervals with length ${\hat{d}_j}$ or greater. Slack
over the interval ${D_k}$ is calculated from exclusively shorter
intervals. Since the demand of the current interval ${D_k}$ does not
influence the slack at ${D_k}$, replacing \ant{j} in \tasks{} only
affects the slack of intervals with length greater than
${D_k}$. Non-preemptive chunk sizes are assigned based on the
available slack, and only those assigned for an interval of length
greater than ${D_k}$ can be affected by replacing \ant{j} in \tasks{}. 
\end{prop}

\begin{algorithm}[ht]
  \caption{\text{\sc{divide}}}
  \label{alg:divide}
  \begin{algorithmic}[1]
    \Procedure{divide}{\ant{j}, ${q}$}
        \State ${\Partial{j} \gets \{\}}$
        \State ${m \gets \underset{m \in \mathbb{Z}^+}\argmax
          \left(\hat{c}_i(m) \le q \right)}$
        \label{line:m-max}
        \State ${r \gets \hat{m}_j}$
        \While {${r > 0}$}
            \State ${m_p \gets \min(r, m)}$
            \State ${\task{p} \gets (\hat{p}_j, \hat{d}_j, m_p,
              \hat{c}_j)}$
            \Comment Posterior task, same period, deadline, WCET function.
            \State ${\Partial{j} \gets \Partial{j} \cup \task{p}}$
            \State ${r \gets r - m_p}$
        \EndWhile
      \State return \Partial{j}
    \EndProcedure
  \end{algorithmic}
\end{algorithm}

On Line~\ref{line:divide} of the \tpj{} Algorithm~\ref{alg:tpj}, the task
\ant{j} is divided into \Partial{j} by the \text{\sc{divide}}
procedure. Pseudocode of \text{\sc{divide}} is given by
Algorithm~\ref{alg:divide}. The number of tasks in \Partial{j} are
determined by the maximum number of threads ${m}$ of \ant{j} that can
execute non-preemptively within ${q}$ time units. Each task ${\task{k}
  \in \Partial{j}}$ is assigned ${m}$ threads of \ant{j} or however
many remain, whichever is less. The result is that each task set
has the following properties. 

\begin{prop}[Partial Task Sets Returned from \texdivide{}]
  \label{prop:partial-task-set-size} The partial task set \Partial{j}
  of an anterior task \ants{} for a specific ${q}$ value (and related
  maximum threads assigned per job ${m}$ such that ${c_j(m) \le
    q}$) contains posterior tasks where:
  \begin{enumerate}
    \item The exact number of posterior tasks is ${|\Partial{j}| = \lceil
      \frac{\hat{m}_j}{m} \rceil}$ 
    \item Exactly ${\lfloor \frac{\hat{m}_j}{m} \rfloor}$ tasks of
      \Partial{j} are assigned ${m}$ threads per job.
    \item There is at most one task ${\task{g} \in \Partial{j}}$ with
      exactly ${m_g = \hat{m}_j \mod m}$ threads.
  \end{enumerate}
  
\end{prop}
  

