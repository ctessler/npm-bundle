\begin{theorem}[Minimal Demand of Partial Task Sets Over All Intervals]
  \label{thm:all-demand}
  For a partial task set \Partial{i} of an anterior task \task{i} with
  ${m_i}$ threads, minimizing ${\sum_{\task{k} \in \Phi_i}
    \dbf{\task{k}}{\deadline{i}}}$ minimizes ${\sum_{\task{k} \in
      \Phi_i} \dbf{\task{k}}{${t}$}}$ for all ${t \ge 0}$.
\begin{proof}
  Provided into two parts, when ${t < \deadline{i}}$ and
  ${t \ge \deadline{i}}$. The first portion is a simple direct
  argument. The second portion is by contradiction.

  \emph{Part 1}: When ${t < \deadline{i}}$,
  ${0 = \sum_{\task{k} \in \Phi_i}\dbf{\task{k}}{$t$}}$. By definition
  of the demand bound function (Equation~\ref{eq:dbf}) the execution
  requirement of a task is zero before the first possible
  deadline. All tasks ${\task{k} \in \Partial{i}}$ share the same
  relative deadlines ${d_k = d_i}$ and absolute deadlines because
  ${p_k = p_i}$. These follow from the definition of
  division (Definition~\ref{def:restrict-division}) and partial tasks
  (Definition~\ref{def:partial-tasks}). Since ${t < \deadline{i}}$, ${\dbf{\task{k}}{$t$} = 0}$ for all ${\task{k} \in
    \Partial{i}}$. Therefore, ${\sum_{\task{k} \in \Phi_i}
    \dbf{\task{k}}{${t}$}}$ will be minimal (exactly zero) when ${t <
    d_i}$, regardless of ${\sum_{\task{k} \in \Phi_i}
    \dbf{\task{k}}{\deadline{i}}}$.

  \emph{Part 2}: When ${t \ge \deadline{i}}$, assume
  ${\sum_{\task{k} \in \Partial{i}} \dbf{\task{k}}{\deadline{i}}}$ is 
  minimal and ${\sum_{\task{k} \in \Partial{i}} \dbf{\task{k}}{${t}$}}$ is
  not minimal. Since all partial tasks ${\task{k} \in \Partial{i}}$ share
  absolute deadlines (as described in Part 1), demand for each task
  \dbf{\task{k}}{$t$} increases only for values of ${t}$ that equal
  absolute deadlines. Furthermore, the execution requirement of
  every \task{k} increases exactly by ${c_k(m_k)}$ for each absolute
  deadline of ${\task{i} = \{ \Deadline{1}, \Deadline{2},
    ... \}}$:
  \begin{equation*}
    \indent
    \begin{split}
      \dbf{\task{k}}{\Deadline{1}} &= 1 \cdot c_k(m_k) \\
      \dbf{\task{k}}{\Deadline{2}} &= 2 \cdot c_k(m_k) \\
      ... \\
      \dbf{\task{k}}{\Deadline{z}} &= z \cdot c_k(m_k)
    \end{split}
  \end{equation*}

  Utilizing Property~\ref{prop:dmnd}, for ${t \ge d_i}$ and
  \Deadline{z}, where \Deadline{z} is the greatest absolute deadline
  of \task{i} less than or equal to ${t}$ (${\Deadline{z} \le t}$): 

  \begin{equation*}
    \indent
    \begin{split}
    \sum_{\task{k} \in \Partial{i}} \dbf{\task{k}}{${t}$} &=
      \sum_{\task{k} \in \Partial{i}} z \cdot
      \dbf{\task{k}}{\deadline{i}} 
      & = z \cdot \sum_{\task{k} \in \Partial{i}} \dbf{\task{k}}{\deadline{i}}
    \end{split}
  \end{equation*}

  Because ${z}$ depends on ${t}$ (and is completely independent of the
  division of the partial task set), if ${\sum_{\task{k} \in \Partial{i}}
    \dbf{\task{k}}{$t$}}$ were not minimal then
  ${\sum_{\task{k} \in \Partial{i}} \dbf{\task{k}}{\deadline{i}}}$ could not
  be minimal, contradicting the assumption.

  Combining Parts 1 and 2, when the demand for the partial tasks of
  \task{i} is minimized for the interval ${d_i}$, the demand of
  partial tasks of \task{i} is minimized for all intervals of 
  length ${t \ge 0}$.  
  
\end{proof}
\end{theorem}

\begin{corollary}[Minimal WCET Sum of \Partial{i} Minimizes Demand
    Over the Interval \deadline{i}]
  \label{corollary:min-demand-di}
  The demand of \Partial{i} over the interval \deadline{i} is
  minimized when the sum of WCET of \Partial{i} is minimized.
\begin{proof} Following directly from
  Theorem~\ref{thm:all-demand}, where the demand over the interval
  \deadline{i} of each task ${\task{k} \in \Partial{i}}$ is given by
  ${\dbf{\task{k}}{\deadline{i}} = 1 \cdot c_k(m_k) =
    c_k(m_k)}$. Then,

  \begin{equation*}
    \indent
    \begin{split}
      \sum_{\task{k} \in \Partial{i}} \dbf{\task{k}}{\deadline{i}} &=
        \sum_{\task{k} \in \Partial{i}} c_k(m_k)
    \end{split}
  \end{equation*}

  Thus, minimizing ${\sum_{\task{k} \in \Partial{i}} c_k(m_k)}$
  minimizes ${\sum_{\task{k} \in \Partial{i}}
  \dbf{\task{k}}{\deadline{i}}}$
\end{proof}

\end{corollary}

\begin{corollary}[Minimal WCET Sum of \Partial{i} Minimizes Demand
    Over all Intervals ${t \ge 0}$]
  \label{corollary:min-wcet-di}
  The demand of \Partial{i} over alls interval ${t \ge 0}$ is
  minimized when the sum of WCET of \Partial{i} is minimized.
\begin{proof} Following directly from
  Theorem~\ref{thm:all-demand} and Corollary~\ref{corollary:min-demand-di}.
\end{proof}

\end{corollary}

