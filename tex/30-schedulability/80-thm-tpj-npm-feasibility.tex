\begin{theorem}[\tpj{} is Optimal with Respect to npm-feasibility]
  \label{thm:tpj-optimal}
  For a task set specification \ants{}, \tpj{} returns feasible if and
  only if there exists an npm-feasible posterior task set \tasks{} of
  \ants{}.

  \begin{proof}
    \emph{Forward Direction} (\tpj{} returns feasible for \ants{}
    ${\implies \exists}$ a posterior task set ${\tasks{}~|~\tasks{}}$
    is npm-feasible): The \tpj{} algorithm returned a 
    posterior task set \tasks{} where the infeasibility condition
    (Equation~\ref{eq:baruah-07}) is never satisfied across intervals
    of length ${0 \le t \le T^*(\tasks{})}$ and every job of
    ${\task{i} \in \tasks{}}$ executes non-preemptively for ${c_i(m_i)
      \le q_i}$ time units. Therefore, \tasks{} is npm-feasible. 

    \emph{Reverse Direction} (${\exists}$ a posterior task set
    ${\tasks{}~|~\tasks{}}$ is npm-feasible ${\implies}$ \tpj{}
    returns feasible for \ants{}): For the purpose of demonstrating a
    contradiction, assume \tpj{} returns infeasible for an
    npm-feasible task set \ants{}. Name the absolute deadline which
    \tpj{} returned infeasibility for ${D_x}$ from the set
    ordered deadlines ${\{D_1, D_2, ... \}}$ and the task which
    generated ${D_x}$, \ant{x}. Name the set of tasks with relative
    deadlines smaller than ${\hat{d}_x}$, ${\bar{\tasks{}}}$. 

    For any task ${\task{k} \in \bar{\tasks{}}}$ and partial task set
    ${\Partial{k}}$ of ${\task{k}}$ included in the posterior set
    \tasks{}, the number of tasks and threads assigned to each
    \Partial{k} cannot be affected by \ant{x} due to
    ${\hat{d}_x > d_k}$ and Property~\ref{prop:divisions}. The
    combined set of posterior tasks of ${\bar{\tasks{}}}$ in \tasks{}
    is denoted
    ${\dot{\tasks} = \cup_{\task{k} \in \bar{\tasks{}}} \Partial{k}}$.
    
    There are two cases where \tpj{} will return infeasible for
    \ants{}, on Line~\ref{line:no-slack-infease} and
    Line~\ref{line:more-demand}. Both illustrate a contradiction with 
    the respect to demand.

    \emph{Line~\ref{line:no-slack-infease}}: If \tpj{} returns
    infeasible for \ants{} on Line~\ref{line:no-slack-infease} there is
    insufficient slack ${q_x}$ to execute any one-thread job of
    \ant{x} non-preemptively. Since slack is inversely related to
    demand, the demand of ${\dot{\tasks{}}}$ is too great to allow any
    thread of \task{x} as part of a feasible task set.

    \emph{Line~\ref{line:more-demand}}: If \tpj{} returns infeasible
    for \ants{} on Line~\ref{line:more-demand}, there is insufficient
    supply for \Partial{x} (the set of partial tasks of \ant{x}). By
    Corollary~\ref{corollary:min-demand-di} and
    Theorem~\ref{thm:tpj-wcet-sum} the demand of \Partial{x} is
    minimal over all intervals for the available slack ${q_x}$. Due to
    Property~\ref{prop:divisions} only tasks with shorter relative
    deadlines i.e. ${\dot{\tasks{}}}$, can impact the demand of
    \Partial{x} by affecting ${q_x}$. In this case, the demand of
    ${\dot{\tasks{}}}$ is too great for the demand of \Partial{x} to
    be included as part of a feasible task set.

    By assumption \ants{} is npm-feasible, the infeasibility
    conditions on Lines~\ref{line:no-slack-infease}
    and~\ref{line:more-demand} of \tpj{} indicate the demand of 
    ${\dot{\tasks{}}}$ is too great. However, \tpj{} adds each
    partial set \Partial{k} to ${\dot{\tasks{}}}$ in increasing
    deadline order. By Property~\ref{prop:divisions}, every 
    \Partial{k} added to ${\dot{\tasks{}}}$ exclusively impacts the
    demand of larger deadlines. Every \Partial{k} increases the demand of
    ${\dot{\tasks{}}}$ minimally starting with ${D_1}$, maximizing the
    slack available for partial task sets with greater deadlines; thus
    the demand of ${\dot{\tasks{}}}$ is minimal and cannot be
    reduced. For \ants{} to be npm-feasible, there must be another
    partial task set that reduces ${\dot{\tasks{}}}$'s demand,
    which is a direct contradiction. Therefore, \tpj{} must return
    feasible.
  \end{proof}
\end{theorem}
