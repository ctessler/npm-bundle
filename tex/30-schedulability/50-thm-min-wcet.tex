
%
% Un-shared assumption version
%
%% \begin{theorem}[Minimal Sum of WCET of \Partial{i} for any ${q}$ by
%%   \texdivide{}]
%%   For an anterior task \task{i} and non-preemptive chunk size ${q}$,
%%   \texdivide{} will produce a partial task set \Partial{i} with
%%   minimum WCET sum among all possible partial task sets of \task{i},
%%   where all jobs of tasks of \Partial{i} complete execution
%%   non-preemptively within ${q}$ time units given the WCET function of
%%   ${c_i(m_i)}$ is a strictly increasing discrete concave function.
%% \end{theorem}

\begin{theorem}[Minimal Sum of WCET of \Partial{i} for any ${q}$ by
    \texdivide{}]
  \label{thm:tpj-wcet-sum}
  For an anterior task \ant{i} and non-preemptive chunk size ${q}$,
  \texdivide{} will produce a partial task set \Partial{i} with
  minimum WCET sum among all possible partial task sets of \ant{i}.

  \begin{proof} To illustrate a contradiction, assume \Partial{i}
    returned from \texdivide{} does not have the minimal WCET sum
    for a specific ${q}$ and task \ant{i}. There must exist  a partial
    task set \Partial{k} of \ant{i} that differs, ie. ${\Partial{i}
      \not = \Partial{k}}$ and 
    \begin{equation*}
      \indent
      \sum_{\task{k} \in \Partial{k}} c_k(m_k) <
      \sum_{\task{j} \in \Partial{i}} c_j(m_j)
    \end{equation*}

    By Property~\ref{prop:partial-task-set-size} of partial tasks
    created by \texdivide{}, \Partial{i} will have at most one task
    with less than ${m}$ threads assigned to it. For \Partial{k} to
    differ, it must have at least two tasks with less than ${m}$
    threads assigned to them. Call these two tasks with less than
    ${m}$ threads ${\task{x}, \task{y} \in \Partial{k}}$. Select
    \task{x} as the task with the greater number of threads
    ${m_x \ge m_y}$.

    Consider the impact on ${\sum_{\task{k} \in \Partial{k}}
      c_k(m_k)}$ of moving one thread of \task{y} to \task{x}, as the
    operation of adding the difference of WCET values for
    ${c_x(m_x + 1)}$ and ${c_y(m_y - 1)}$ to the sum. 
    
    \begin{eqnarray*}
      \lefteqn{
      \left(\sum_{\task{k} \in \Partial{k}} c_k(m_k) \right)
      - c_x(m_x) + c_x(m_x + 1)
      - c_y(m_y) + c_y(m_y - 1)} \\
      & = & \left(\sum_{\task{k} \in \Partial{k}} c_k(m_k) \right)
      + (c_x(m_x + 1) - c_x(m_x)) - (c_y(m_y) - c_y(m_y - 1))
    \end{eqnarray*}
    
    By the concave growth Property~\ref{prop:ni-growth} and virtue of
    ${m_y \le m_x}$, the quantity ${(c_x(m_x + 1) - c_x(m_x))}$ is
    less than or equal to ${(c_y(m_y) - c_y(m_y - 1))}$ so the
    difference must be less than or equal to zero. Therefore:

    \begin{equation*}
      \left(\sum_{\task{k} \in \Partial{k}} c_k(m_k) \right)
      + (c_x(m_x + 1) - c_x(m_x)) - (c_y(m_y) - c_y(m_y - 1)) \le 
      \sum_{\task{k} \in \Partial{k}} c_k(m_k)
    \end{equation*}

    The WCET sum of \Partial{k} can be reduced by moving one thread of
    \task{y} to \task{x}. When ${m_x = m}$ no more threads may be
    assigned to \task{x} or the system will be infeasible by
    Definition~\ref{def:assumptions}. While there are two (or more)
    tasks of ${\task{x}, \task{y} \in \Partial{k}}$ with fewer than
    ${m}$ threads assigned, moving one thread from \task{y} to
    \task{x} will reduce the WCET sum. By repeatedly moving tasks to
    reduce the WCET sum, \Partial{k} will satisfy all aspects of
    Property~\ref{prop:partial-task-set-size} of partial task sets
    created by \texdivide{}, ie. ${\Partial{i} = \Partial{k}}$ after
    all moves have been completed. This contradicts the assumption
    that ${\Partial{i} \not = \Partial{k}}$ and the relationship of
    their WCET sums, therefor \Partial{i} is minimal.
  \end{proof}
\end{theorem}
