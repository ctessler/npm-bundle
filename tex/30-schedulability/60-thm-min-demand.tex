\begin{theorem}[\tpj{} Minimizes the Demand of a Task Over All
  Intervals for Feasible Task Sets]
  \label{thm:tpj-min-demand}
  For a task \ant{i} in the anterior task set \ants{}, with a
  strictly increasing discrete concave WCET function ${c_i(m_i)}$, \tpj{}
  minimizes the total demand of the partial set of tasks \Partial{i},
  ${\sum_{\task{k} \in \Partial{i}} \dbf{\task{k}}{$t$}}$ for ${t \ge 0}$
  in the feasible posterior task set \tasks{}.

  \begin{proof} There are two cases for a task
    ${\ant{i} \in \ants{}}$. In each case, the WCET sum of the partial set
    \Partial{i} is minimized. The minimal WCET sum is then related to
    demand by Corollary~\ref{corollary:min-demand-di}. The number of
    threads for an anterior task ${\ant{i}}$ is referred to as
    ${\hat{m}_i}$, and ${m_i}$ in the posterior task \task{i}.

    \emph{Case 1} (${c_i(\hat{m}_i) \le q_i}$): \ant{i} has a WCET
    value smaller than any non-preemptive chunk size calculated for
    \ant{i}. If this is the case, \ant{i} is not divided 
    in \tasks{} and ${\Partial{i} = \{\task{i}\}}$ where
    ${\task{i} = \ant{i}}$.
    Due to the concavity of ${c_i(m)}$ the WCET of ${\hat{m}_i}$ is
    less than WCET sum of any partial set \Partial{j} of task \ant{i}:
    ${c_i(\hat{m}_i) \le \sum_{\task{k} \in \Partial{j}} c_k(m_k)}$.

    \emph{Case 2} (${c_i(m_i) > q_i}$): \ant{i} has a WCET value
    greater than some non-preemptive chunk size calculated for
    \ant{i}. Chunk sizes are calculated once for the task 
    \ant{i}, when ${D_k = d_i}$ and \ant{i} is divided
    according to the \texdivide{} algorithm, producing the minimal
    WCET sum ${\sum_{\task{k} \in \Partial{i}} c_k(m_k)}$ according to
    Theorem~\ref{thm:tpj-wcet-sum}.


    For both cases, the sum of WCET values for the partial set
    ${\sum_{\task{k} \in \Partial{i}} c_k(m_k)}$ is minimized over all
    possible partial sets. Being minimal, the sum satisfies the conditions of
    Corallary~\ref{corollary:min-demand-di}, minimizing the demand  
    over the interval equal to the relative deadline
    ${\sum_{\task{k} \in \Partial{i}} \dbf{\task{k}}{$d_i$}}$,
    and all intervals
    ${\sum_{\task{k} \in \Partial{i}} \dbf{\task{k}}{$t$}}$
    by Theorem~\ref{thm:all-demand}.
  \end{proof}

  \begin{corollary}[\tpj{} Minimizes Demand of All Tasks for Feasible
      Task Sets]
    \label{corollary:tpj-min-demand-all-tasks}
    For an anterior task set \ants{}, where each task ${\ant{i} \in \ants}$
    has a strictly increasing discrete concave WCET function
    ${\hat{c}_i(\hat{m}_i)}$, \tpj{} minimizes the total demand of the
    posterior task set \tasks{}, ${\dbf{\tasks}{$t$}}$ for ${t \ge 0}$
    if the posterior task set is feasible.

    \begin{proof}
      By Theorem~\ref{thm:tpj-min-demand}, the demand of each partial
      set generated by any ${\ant{i} \in \ants{}}$ is minimal over all
      intervals. The posterior task set \tasks{}, is the union of the
      partial sets generated from \ants{}. Thus, the demand of
      \tasks{} is minimal over any interval due to the contributions
      of each partial set being minimal over any interval.
    \end{proof}
    
  \end{corollary}
\end{theorem}
