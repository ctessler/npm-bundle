In the non-preemptive context and schedulability test given by Threads
Per Job, the \emph{optimal} division of a task into a partial task set
is the division which maximizes the number of feasible task sets. 

\begin{theorem}[TPJ is Optimal]
  For task set \tasks{} where all tasks ${\task{i} \in \tasks{}}$
  have non-increasing growth, TPJ divides tasks optimally. 

  \begin{proof} By Theorem~\ref{thm:tpj-min-demand}, TPJ produces the
    minimal demand over any interval for all tasks according to the
    non-preemptive assigned. Infeasibility is tested on
    Lines~\ref{line:infeasible-1} and \ref{line:infeasible-2} of
    Algorithm~\ref{alg:tpj}.

    Both infeasibility tests examine the available slack. If there is
    too little slack, the task set is infeasible. Since slack is
    dependent upon demand, producing the minimal demand maximizes the
    available slack. TPJ minimizes demand over any interval,
    maximizing slack in any interval. This maximizes the number of
    feasible task sets, therefor TPJ is optimal.%
  \end{proof}
\end{theorem}
