\section{Evaluation}\label{sec:eval}
\renewcommand{\tex}[1]{70-evaluation/#1}

Evaluation~\cite{NPM-Artifact:2019} of \tpj{} and the non-preemptive
multi-threaded task model presented in this work focuses on the
schedulability ratio of synthetic task sets and a case study based
upon the evaluation of \bundlep{}~\cite{Tessler:2018b}. The ratio of
task set specifications deemed schedulable by \tpj{} for EDF-NP will
be compared to \npchunks{} in both limited and fully preemptive
settings for EDF. What follows is a description of the parameters to
task set specification generation, the prescribed evaluation metrics,
and analysis of the results. 

\subsection{Generating Task Sets}

A specified task set \tasks{} is generated with four parameters,
${M}$ the total number of threads of execution, ${U}$ the target
utilization, a maximum growth factor \GFactor{}, and ${m}$ the maximum
number of threads per task. The number of threads ${M}$ may be one of
${\{3, 5, 7, 10, 25, 50, 100\}}$ with dependent ${m}$ values of
${\{2, 2, 3, 4, 8, 16, 32\}}$. Utilization varies from [0.1, 0.9]
and the growth factor varies from [0.1, 0.9] independently by
increments of 0.1. 

Each task ${\task{i} \in \tasks{}}$ is assigned ${m_i}$ threads from
a random uniform integer distribution over ${[1, m]}$, such that the
sum of all threads is equal to ${M = \sum_{\task{i} \in \tasks{}}
  m_i}$. A task's period \period{i} is from a uniform integer
distribution over [10, 1000]. Utilization ${u_i}$ of each task
\task{i} is calculated using the ${\text{UUniFast}(n,
  U)}$~\cite{Bini:2004} algorithm, where ${n = |\tasks{}|}$.

\input{\tex{10-alternate-parameters.tex}}

A task's WCET is assigned for ${m_i}$ threads,
${c_i(m_i) = \lceil p_i \cdot U_i \rceil}$. Tasks
are given a growth factor ${\GFactor{}_i}$ in a uniform real
distribution over ${[0.1, \GFactor{}]}$.  The remaining ${m_i - 1}$
WCET values are determined by substituting ${\GFactor{}_i}$ into
Equation~\ref{eq:gfactor-eval}. The relative deadline of \task{i},
\deadline{i} is taken from a uniform integer distribution over
${[\max(c_i(m_i),p_i/2), 1000]}$.

For each combination of ${(M, m, U, \GFactor{})}$, 1000 task sets
specifications are generated. Figure~\ref{fig:gen-params} summarizes
the parameters of task set generation. The smaller values of ${M}$ are taken
from~\cite{Bertogna:2010} and the dependent ${m}$ values were selected
to avoid one task consuming more than half of the threads in the
task set specification (where possible). 

\subsubsection{Applicability of Parameters}

To avoid favoring \tpj{}, the task set generation parameters ${m}$ and
\GFactor{} were carefully selected. For the threads per task ${m}$, a
large ${m}$ favors \tpj{}. Therefore, no single task my be
assigned more than half the total threads: ${m \le \lfloor \frac{M}{2}
\rfloor}$ (except for ${M = 3}$). 

The growth factor \GFactor{} is informed by previous results for
\bundlep{}~\cite{Tessler:2018b}. In~\cite{Tessler:2018b},
multi-threaded tasks are constructed from the M{\"a}lardalen WCET
benchmarks~\cite{Gustafsson:2010}. Task analysis
in~\cite{Tessler:2018b} yields growth factors below 0.1 for several
benchmarks. A lower bound (0.1) on \GFactor{} greater than observed
values is pessimistic, resulting in less favorable results for \tpj{}.

\subsection{Case Study}

\bundlep{}'s evaluation covers 18 benchmarks for distinct architecture
configurations. An architecture configuration includes the block
reload time (BRT), cycles per instruction (CPI), and number of cache
lines. One of the least favorable in terms of the analytical
benefit of \bundlep{} is a BRT of 100, CPI of one, and 32 cache
lines. From this configuration, the WCET values and growth factors
were extracted, growth factors ranging in the range
${[0.08,3.02]}$.\footnote{Due to length restrictions the full listing
  of WECT and growth factors are omitted.}

From these results of \bundlep{} 1000 task sets with 18 tasks (one per
benchmark) and a total 100 threads were generated per utilization
target. The utilization target ranged from 0.1 to 1.0 increments of
0.1. Threads were assigned to each task \task{i} from a distribution
over ${m_i \in [2,8]}$. Each tasks utilization, period, and deadline,
${c_i(m_i)}$ were assigned using the same method as synthetic tasks.
The WCET values for fewer threads ${1 \le k < m_i}$, were scaled such
that the value of ${c_i(k) / c_i(m_i)}$ remained constant after the
${c_i(m_i) = \lceil p_i \cdot U_i \rceil}$ assignment.


\subsection{Evaluation Metrics}

\tpj{} is compared with the \npchunks{} schedulability test in
non-preemptive (EDF-NP) and preemptive (EDF-P) settings. The focus of
the evaluation is on the non-preemptive setting. The preemptive
setting serves as a comparison to alternative scheduling strategies
and the theoretical best case. For EDF-P, preemptions incur
\textbf{no} penalty, CRPD or otherwise. In this highly advantageous
setting for EDF-P, \tpj{} can still produce feasible non-preemptive task
sets \npchunks{} deems infeasible in a preemptive setting!

To compare schedulability tests, each task set specification \ants{}
is provided to \tpj{} without modification under EDF-NP
scheduling. \tpj{} will transform the task set producing a
posterior task set \tasks{} if a feasible one exists. A task set
specification \ants{} cannot be provided directly to \npchunks{},
since \npchunks{} has no concept of threads per job. 

To be suitable for analysis by \npchunks{}, a task set specification
\ants{} is transformed into two posterior task sets. The first task
set, ${\tasks{}^1}$ represents single-threaded tasks by including all
threads of \ants{} as individual tasks. The second task set,
${\tasks{}^m}$ represents the tasks of \ants{} as indivisible,
executing all specified threads without preemption per job. Each task
in ${\tasks{}^m}$ benefits from the  thread-level scheduler but does
not expose the threaded nature of the task to the scheduling
algorithm. This is achieved by modifying an anterior task \ant{j} with
${\hat{m}_j > 1}$ and ${\hat{c}_j(\hat{m}_j)}$ to a posterior task
\task{j} with ${m_j = 1}$ and ${c_j(1) = \hat{c}_j(\hat{m}_j)}$. 

\begin{wraptable}[7]{l}{.55\linewidth}
%  \begin{table}
  \center
  \begin{tabular}{|c|c|l|l|}
    \hline
    \textbf{Test} & \textbf{Task Set} & \textbf{EDF-NP} & \textbf{EDF-P} \\
    \hline
    \hline
    \tpj{} & \ants{} & EDF-TPJ & - \\
    \hline
    \multirow{2}{*}{\npchunks{}}
    & ${\tasks{}^1}$ & EDF-NP:1 & EDF-P:1 \\
    & ${\tasks{}^m}$ & EDF-NP:M & EDF-P:M \\
    \hline
  \end{tabular}
  \caption{Schedulability Test Combinations}
  \label{table:combinations}
%  \end{table}
\end{wraptable}

The \npchunks{} schedulability test will produce results for
${\tasks{}^1}$ and ${\tasks{}^m}$ in both preemptive and
non-preemptive settings. For non-preemptive schedulability analysis,
each task ${\task{i} \in \tasks{}^1}$ or ${\tasks{}^m}$ must have a
non-preemptive chunk size ${q_i \ge c_i(m_i)}$. When evaluating
preemptive EDF schedulability for ${\tasks{}^1}$ and ${\tasks{}^m}$,
the results are labeled EDF-P:1 and EDF-P:M respectively. When
evaluating non-preemptive EDF schedulability, the results are labeled
EDF-NP:1 and EDF-NP:M. Schedulability results for \tpj{} under EDF-NP
scheduling are labeled EDF-TPJ. Table~\ref{table:combinations} gives a
synopsis of the schedulability tests. Schedulability ratios for each
of the combinations are calculated for every ${(M, m, U, \GFactor{})}$ 
configuration.


It must be noted that EDF-P:M is an unrealistic schedulability
test. It serves only as a theoretical limit to the benefits of concave
growth. Concave growth is a result of scheduling threads of the same
job without preemption by another job with a \texttt{BUNDLE}-based
thread-level scheduler.  However, current \texttt{BUNDLE}
implementations require that an executing task cannot be
preempted by a different task.  Such a preemption would destroy the
cache benefits and analysis of \texttt{BUNDLE} scheduling. Analysis of
EDF-P:M assumes preemptions between jobs are allowed and
have zero cost. It is included as a reference for \tpj{}'s
performance, as a ceiling for what is theoretically possible given
ideal (but likely impossible) conditions.

As a consequence of transforming multi-threaded task set
specifications \ants{} to single-threaded task sets ${\tasks{}^1}$,
some single threaded task sets may not be feasible. One reason for a
task set ${\tasks{}^1}$ to become infeasible is the 
utilization exceeding one, while ${\tasks{}^m}$ and \ants{} have
utilization less than one. In this setting, EDF-TPJ
is capable of scheduling task sets that preemptive EDF cannot.

For a task set specification configuration ${(M, m, U, \GFactor{})}$,
call ${S}$ the set of all task set specifications \ants{} generated
for the configuration. Call ${s}$ the set of ${\tasks{}^1}$ task sets
transformed from ${\ants{} \in S}$ such that ${\tasks{}^1}$ has
utilization greater than one. The set \stpj{} is the subset of ${s}$ deemed feasible
by the \tpj{} schedulability test. That is, \stpj{} is the set of all
tasks \tpj{} could schedule, yet EDF-P:1 could not (even) when CRPD
values are zero.

\input{\tex{30-results.tex}}
